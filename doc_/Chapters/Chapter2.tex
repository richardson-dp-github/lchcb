% Chapter Template

\chapter{Related Work} % Main chapter title

\label{Chapter2}
\label{Related Work}% Change X to a consecutive number; for referencing this chapter elsewhere, use \ref{ChapterX}

\section{Cassandra}
%(move from Chapter 1 to here)
Cassandra is a widely used distributed \gls{nosql} database with many use cases \cite{Lakshman2010CassandraSystem}.
Not only has Cassandra been reportedly been used in practice \cite{ApacheCases}, but has been, using the Yahoo Cloud Services Benchmark \cite{YahooBenchmark}, formally evaluated in scholarly literature against other databases such as MongoDB and proposed as the NoSQL database of choice in the Internet of Things and distributed sensor networks \cite{Abramova2013NoSQLCassandra}.
There have been credible claims of Cassandra being used on Raspberry Pi \cite{VanRyswykMulti-DatacenterPis,SercelCassandraMedium}, but to the author's knowledge, no white paper with the details exists.
%But in a way, this is the simplified version of the application that this is testing.
The aim of this paper is to examine Cassandra's performance coupled with a simplified Wi-Fi collection and analysis application, where the nature of link nodes may be less reliable than wired Ethernet.

This author's interest in Cassandra lies in the fact that Cassandra is a distributed database used in practice for cloud computing.
Although it describes itself as a NoSQL database, the interface allows for SQL commands and has a Python API.
Cassandra allows for configuration of distributed systems parameters, such as replication factor, but detailed knowledge of distributed systems protocols is not critical for operation.

From an experimental standpoint, the distributed nature of a Cassandra "keyspace" lies in four parameters \cite{CassandraDummies}: cluster size (the number of nodes), replication factor (configured in software), write level (configured in software), and read level (configured in software).

\section{Raspberry Pi 2}
%(move from Chapter 1 to here)
The Raspberry Pi 2 (Model B) \cite{RaspberryB} is a low-cost computer designed sold from the United Kingdom.
It can be described as a motherboard for a about the size of a 3x5 index card and has been available since February 2015.
This experiment is interested in the Raspberry Pi 2 as a representative of the low-cost hardware domain, which implies low cost, low power consumption, and low in terms of size and weight.

This author's interest in the Raspberry Pi 2 is that its ARM Cortex-A7 processor and 1GB RAM \cite{RaspberryB} makes it a key representative of the low-cost hardware domain and the Internet of Things.
The Raspberry Pi 2 has cost as low as 35 USD \cite{RaspberryPi}.
It is lightweight and has limited power consumption \cite{RaspberryB}.
Constraints on size, weight, power, and cost can all be barriers to entry for applications seeking computing nodes.

Expanding this experiment, to say the BeagleBone black \cite{BeagleBoard.orgBlack}, is in touch with the spirit of this experiment but outside the scope of this paper.
See section \ref{Conclusion} for elaboration on future work.

%2.3          Small Cluster Computing 
\section{Small Cluster Computing}
%Baun
Considering the educational purpose of Raspberry Pis, it is no surprise to find academic interest in Raspberry Pi clusters.
For instance, \cite{Baun2016MobileResearchers.} illustrates some variation in SD Card performance.
This paper keeps the \gls{sd} Card class constant, but knowing that the type of SD card used tempers the effects.
Baun highlights the limits of the SD Card controller, which in turn highlights the potential value of testing actual hardware in addition to virtual machines, where there may be some variation in the limits hard-disk controllers.

%Small Data Centers
Existing Raspberry Pi clusters, built to serve as a "practical balance" \cite{Tso2013TheInfrastructures}, such as in \cite{Kiepert2013CreatingCluster} and \cite{Tso2013TheInfrastructures}, suggest the value of Raspberry Pi nodes compared to large, traditional servers both in terms of power construction and actual purchasing price.
In \cite{Velthuis2015SmallStreaming}, Cassandra is used to store videos for a video streaming application on after a "a lot of configuration", albeit the configuration parameters were unspecified.
However, this paper, \cite{Velthuis2015SmallStreaming} at least shows a high index of suspicion that Cassandra can be used in a small cluster environment.
%Strengths
%Weaknesses

%2.4          Benchmarking Distributed Databases  - Benchmarks from Chapter 1 goes here as well 
\section{Benchmarking Distributed Databases}
\subsection{Benchmarks}

Benchmarks are the common parlance for a way to test a computing system's capabilities, whether that be.  
For instance 
A curious reader may view a slew of computing benchmarks at the Wikipedia page \cite{Category:ComputerEncyclopedia}.
In this experiment, the "cassandra-stress" \cite{DatastaxTheTool} tool is used.
The "cassandra-stress" tool, used in previous benchmarking efforts such as the one by John Sercel \cite{SercelCassandraMedium}, provides for natural.
A custom benchmark, while flexible, can open up a Pandora's box of holes and inconsistencies, some that may never even come across the developer's mind.
The effort toward a custom benchmark was suspended by this author.

There has been a lot of interest in testing distributed databases, databases that cover multiple nodes.
Paper \cite{Cooper2010BenchmarkingYCSB}, presents the \gls{ycsb}, highlighting "scaleup" and "elastic speedup" as parameters for benchmarking.
%Strengths
It provides a survey of five databases: PNUTS, BigTable, HBase, Cassandra, and Sharded MySQL.
Although Cassandra comes with its own stressor application, cassandra-stress, this particular software is unable to test other distributed databases if a comparison is desired.
%Weaknesses
As might be expected, Cassandra has the ability to be tuned based on the application, data distribution, workload type, etc.
In \cite{Cooper2010BenchmarkingYCSB}, they claimed to "[tune] each system as best [they] know how."
In contrast, this paper will attempt to identify any tuning parameters that have been modified from the default.
It is also worth noting that the version of Cassandra has evolved from year 2010, the time \cite{Cooper2010BenchmarkingYCSB} was published.

%Part of this paper's methodology takes after the methodology in \cite{Abramova2014TestingCassandra} and takes it a bit further, trying to get a sense of how Cassandra performs on Raspberry Pi.
Cassandra was shown in \cite{Abramova2013NoSQLCassandra} to be favorable to write-heavy workloads compared to another database in the domain.
%Although the \gls{iot} implies write-heavy workloads, both \cite{Abramova2014TestingCassandra}
%Strengths
Notable about this paper is that the paper scales the node's \gls{ram} down to 2GB, compared to higher powered machines in other papers such as the Cooper paper \cite{Cooper2010BenchmarkingYCSB} or as specified on the website \cite{CassandraHardwareWiki}.
Although it is not explicitly mentioned as an interest in the paper, this shows a transition of using Cassandra for lower powered machines.
%Weaknesses
One thing that is not clear in \cite{Abramova2014TestingCassandra} is how cache effects are accounted for.
If unaccounted for, a cache effect may result in the initial run resulting in longer execution times than subsequent runs, all other factors being constant. 
The key cache is set at 100 MB and the row cache at 0.
In contrast, this paper clears the data from (or truncates) the table of interest.
\cite{Abramova2014TestingCassandra} mention that each 7200 rpm with no stated limits on hard drive space.
Moving into the realm of in-situ storage, this paper takes a significant deviation in limiting the hard disk space to 8 or 16 GB.
%Other Notes
%2.5          Evaluating Potential Uses
\section{Evaluating Potential Uses}
%WiFi Collection (move from Chapter 1 to here)
%WiFi Mapping (move from Chapter 1 to here)
\subsection{WiFi Mapping}
There has been much work done with respect to WiFi mapping.  Argos \cite{Rose2010MappingArgos} describes a similar system of a distributed system, but there is no mention of Cassandra or any distributed database, which may serve as an improvement on such a sensor network.  Wigle.net \cite{WiGLE:Mapping} is an aggregate map that distributes a smart-phone application to collect GPS coordinate-Access Point pairs, but relies on a central database, and has an unreliable user base and irregular sampling frequencies (relying on the public).  It is also worth noting that records are not updated.  Heat Mapper \cite{HeatMapperOffices} is partially free and commercial software that can generate a heat map for a small room or office.  Wi2Me \cite{Castignani2012Wi2Me:Networks} performs this mapping as well, with an emphasis on performance and data throughput.  It uses an instance of SQLite to store the traces on the individual's smart-phone, but again, none of these make use of a distributed database like Cassandra as part of the sensor network.

\subsection{WiFi/Wireless Crowd Detection}
There are numerous blog posts that lay claim to the fact that individuals are tracked via commerical entities via WiFi \cite{HaighTrackingConnected}. Some have reported to make art exploiting this mechanism \cite{KeebleCasual2013}.

There have been efforts to track crowds, notably \cite{Bonne2013WiFiPi:Events} and \cite{Schauer2014EstimatingBluetooth}.  

\subsection{\gls{cbir}}

\gls{cbir} refers to a way of querying images based on derived features.
There are numerous works with regard to \gls{cbir}.
Applying this is another way to put the lower-grade hardware to the test.

\subsection{Other}

%2.6          Research Trends and Gaps - 3-5 sentences that alludes to the lack of key 

\section{Research Trends and Gaps}

Distributed databases have been around for a long time, but hardware like the Raspberry Pi presents a new suite of known or yet to be developed applications that desire a distributed database at its service.
Configuring Cassandra for limited disk space is an area that is commonly left out.  This makes sense, given that in general disk storage is relatively cheap.  Configuring as such requires some extra configuration, but it should be noted that as distributed databases like Cassandra develop, these things may automatically be configured.
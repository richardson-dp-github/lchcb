\chapter{Case Studies}

\label{Case Studies}
\label{Chapter5}
%(CONTRIBUTION 2 –Implement and evaluate target applications using a distributed NoSQL database in IoT)
%5.1          Overview 
%5.2          Wifi Collection, Mapping and Crowd Detection
                Overview, application purpose
%5.2.1      Data Schema
%5.2.2      Implementation – WiFiPi prototype
%5.2.3      Experimental results
%5.2.4      Discussion of results
%5.3          CBIR
%5.3.1      Data Schema
%5.3.2      Implementation – MSACS
%.3.3      Experimental results
%5%.3.4      Discussion of results
%5.4          Summary of Findings for Applications

\section{Overview}
It is expected that within the realm of \gls{iot}, resources will be shared between storing data and other functions upon a given computing node.
This chapter narrates how the testing of Cassandra relates to fitting such a distributed database into various applications.

\section{Wifi Collection, Mapping and Crowd Detection Overview, application purpose}
Given the ubiquity of 802.11 traffic, it is often desirable to gauge one's environment.
For example, mapping applications may supplement additional navigational tools.
Or, WiFi collection may give an idea of crowds and WiFi usage.

\subsection{Data Schema}
The structure for a data schema is as simple as:
\begin{enumerate}
\item Unique ID
\item Timestamp
\item Advertised \gls{mac} Address
\item \gls{ssid}
\end{enumerate}

\subsection{Implementation of WiFiPi prototype}

The WiFiPi prototype consisted of three nodes.  Each node contained a WiFi traffic sensor plus a second wireless interface for internode communication.




\subsection{Experimental results}
Insert graph here:

Performance latency
\subsection{Discussion of results}
\section{\gls{cbir}}
\subsection{Data Schema}
\subsection{Implementation – \gls{msacs}}
\subsection{Experimental results}
\subsection{Discussion of results}
\section{Summary of Findings for Applications}
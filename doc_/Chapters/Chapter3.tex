% Chapter Template

\chapter{Cassandra Benchmark Methodology} % Main chapter title

\label{Methodology}
\label{Chapter3}

\section{Objective and Outline of Methodology}

The objective of these experiments is to get a sense of how sensitive a given distributed database, Cassandra, is to less capable hardware.


%\section{Overview}

%Upon download, Cassandra comes with a stress testing tool, 'cassandra-stress' \cite{DatastaxTheTool} which can be configured to write and read different keyspaces and adjust their parameters based on a configuration (.yaml) file.





%3.2          Sensitivity Testing – explanation of terms, some preliminary studies on the impact of varying parameters/configuration
\section{Sensitivity Testing}

Benchmarking or testing the network often keeps many tunable parameters constant.
One may want to have an idea which of these parameters result in coarse, fine, or virtually no tuning at all.
Some parameters, like replication, have a direct impact on key performance parameters: latency, throughput in operations per second.

%3.3          Overview of Experiments
\section{Overview of Experiments}

This methodology is adapted from \cite{Abramova2014TestingCassandra} to determine the end of the 
The Yahoo Cloud Services Benchmark \cite{YahooBenchmark} is a commonly used benchmarking tool, and is used in \cite{Abramova2013NoSQLCassandra} and \cite{Abramova2014TestingCassandra}.

This experiment varies the parameters in the configuration (.yaml) file and gauge the sensitivity in performance parameters: operations per second and latency.
Cassandra does come with its own stress tool, cassandra-stress, but the Yahoo Cloud Services Benchmark \cite{YahooBenchmark} is preferred due since it can be used to benchmark a multitude of competing distributed databases in possible future work.
To gauge scalability the number of nodes will be varied between 1, 3, and 6.
The total size of the data will be varied between 500k, 800k, and 1M records.  This is adapted from \cite{Abramova2014TestingCassandra} which varies between 1M, 10M, and 100M records, and serves to illustrate the expected variance between reading from large and small databases.  Although a more robust comparison might be desirable, storage limits come into play.
The last thing that will be varied on a given configuration is workload: standard \gls{ycsb} workloads A, C, and E.  This will illustrate a sense of the variance that can ensue as a result of application variance.

The nature of nodes will be varied as follows: virtual machine nodes with 2 GB \gls{ram}, virtual machine nodes with 1 GB \gls{ram}, actual Raspberry Pi 2 hardware nodes with 1 \gls{ram} \cite{RaspberryPi} over a wired ethernet \gls{lan}, and 

%3.3.1      Bandwidth testing
\subsection{Bandwidth Testing}

%3.3.2      Platform testing
\subsection{Platform Testing}

Even if an exact \gls{bom} is available for a given piece of hardware, the granularity with which a distributed database's performance can be simulated in a virtual environment may be limited.

\subsubsection{Virtual Machine Setup Details}

The host machine used was 
For setting up the virtual machine, the 
VirtualBox was downloaded.  The 
VirtualBox Graphical User Interface Version 5.0.24\_Ubuntu r108355

%3.3.3      Network configuration testing
\subsection{Network Configuration Testing}

This methodology will vary the nature of the links from the degenerate case of a nodal network (virtual machines) to wired and wireless \gls{lan}s which imply longer propagation delays and presumably proportional relationship in execution time.

%3.4          Assumptions and Restrictions
\section{Assumptions and Restrictions}




%\section{System Boundaries}



%\section{Experimental Limitations}

%There is a limit to how many 


\subsection{Assumptions}
\label{Assumptions}

Naturally, in order to perform the experiment and evaluate the results, some assumptions had to be made.
Investigation into any of these assumptions may be a lead into future work.

\begin{enumerate}

\item The benchmark represents the application, which assumes a simple schema.  In other words, Cassandra's performance is not particularly sensitive to the schema.

\item There is no active attacker or intrusion into the local area networks.  Both the local area networks are isolated.

% \item There are no errors with the custom benchmark that would skew the results.  Any error is due to the fact that the system's limits have been reached.

\item There are no bugs in the benchmark that would skew the results.

\item Effects on the network due to distance are negligible.



\item This experiment assumes that nodes are homogeneous.  The basis for this assumption is that all nodes have been specified to the same model of Raspberry Pi 2.  The same make and model for the SD Cards have been used.  The image upon the SD Cards has been copied and only adjusted to account for specific, differentiated IP addresses.

% \item This experiment assumes an uninterrupted power supply.  Power supply has no bearing on Cassandra’s performance on the hardware cluster, and is not measured nor accounted for in the model.

\item This experiment assumes an uninterrupted power supply.  Power is not measured nor accounted for in the model.  As long as the power has been turned on, it stays on, and fluctuations in voltage or any kind of imperfections in the power supply are negligible with respect to Cassandra's performance.

\item Although the ISM band is unregulated, this experiment assumes invariant interference from other emitters.  The experiment assumes an urban to suburban environment.  In other words, congestion that overwhelms Cassandra's performance can be assumed to be rare with respect to the population, and is ignored for the purposes of the experiment.

\item Another note on interference for the wireless configurations.  This experiment assumes invariant interference regardless of how much data is being written to the table in Cassandra. For the given pet application, collecting WiFi signals, the benchmark may be related to the amount of expected interference.  Probe requests can be indicative of WiFi traffic with which Cassandra might be competing, but this would require further investigation.  

\end{enumerate}

%\section{Factors Held Constant}
%\label{Factors Held Constant}

%For this experiment, the end of the 


%\section{Response Variables}
%\section{Control Variables}
%\section{Nuisance Factors}
%\label{Nuisance Factors}
%\section{Known/Suspected Interactions}

\subsection{Restrictions}

There are a few settings that had to be taken into account.
Denoted $commitlog\_total\_space\_in\_mb$, this will accumulate to an undesired level.
This was set to 512 MB in the Cassandra configuration file, cassandra.yaml.
Also, to prevent the accumulation of space, the setting $auto\_snapshot$ to false in cassandra.yaml.

Cassandra does run on Java, which may imply some other restrictions.
For example \cite{DatastaxTuningResources} cites the possible restriction of $MAX\_HEAP\_SIZE$ to half of system memory. 


\section{Expected Results}
%               Conclusion paragraph

As the number of nodes increases, the burden of storage operations is expected to decrease in proportion, but the network traffic required to coordinate replication, consistency, and other desirable features may start to increase after some threshold.
There should also a be a proportional increase with the nature of the links.  As the links go from virtual to wired to wireless, the propagation delay will increase, and thus the execution time for a given benchmark will increase in proportion to the propagation delay of the links.

%\section{Design Preferences}

%\section{Analysis and Presentation Techniques}

%The following first-order model will be used, adapted from \cite{Montgomery2013DesignEdition}:

%\begin{equation}
%y=\beta_0+\beta_1x_1+\epsilon
%\end{equation}

%where $y$ is the response variable, $\beta_0$ and $\beta_1$ are the parameters for the first-order model, and $x_1$ represents the quantity of data entries.  

%\section{Coordination}


%\subsection{Ethernet Local Area Network}

%No coordination is needed.
%This network is physically isolated and will not interfere or be interfered with any extra-experiment Ethernet traffic.

%\subsection{Wireless, Ad-Hoc and Local Area Network}
%Because the scale of this experiment was limited, and timing, there really was no formal coordination needed.
%However, because a larger experiment could possibly fill up a frequency channel, one has to be courteous of the environment.
%An extended test would not be appropriate in an uncontrolled environment.

%\subsection{\gls{gsm}}

%Here a special \gls{gsm} lab is needed.
%This required coordination with the Air Force Institute of Technology's Center for Cyberspace Research.

%\section{Simulation Environment}

%In this case, while the benchmark is a simulation of sorts, the focus was on the hardware and how well it would perform.

%\section{Simulation Control Variables}

%What is controlled here is the schema for the
%Cassandra-stress allows one to generate a default benchmark schema or customize one.
%In this case, instead of customizing a schema, the default one is used.


%\section{Treatments}

%\subsection{Type of Links}

%The topologies of interest are wireless topologies, but it is of interest to compare against a comparable Ethernet network.
%A single router of reasonable price may have a limited number of Ethernet cables that can bet attached.
%Cable management can be a burden to a user.
%Wireless topologies allow for more freedom of movement than Ethernet cables.





%\subsection{Nodes and Hashing}

%Cassandra takes advantage of the concept of virtual nodes, which is a software approach to evenly distributing the storage load among nodes no matter what the data hashes into.
%The user can specify the number of tokens, but the default being 256 \cite{DatastaxPartitioners}.
%This experiment will try and determine whether there is an effect on the latency when deviating from the default.
%It is expected that performance will either stay the same or go down as one deviates from the default value, 256.
%At the low extreme, 1, each node is only available once.

%This will be a modified characterization experiment.

%This can be done by creating several different configuration files for running Cassandra.


%\subsection{Caching}

%In Cassandra, a user can choose from two mutually exclusive cache strategies: the partition key cache and the row cache \cite{ConfiguringCaches}.
%Normally, the key cache parameters are maximized.

%This experiment will vary the key cache size.
%The key cache size, parameter $key\_cache\_size\_in\_mb$ is normally set to to 5\% of the available memory or 100MB, whichever is smaller \cite{ConfiguringCaches}.
%This experiment will vary this from 0\% (no cache) to 10\%.

%There is another parameter related to cache, $key\_cache\_keys\_to\_save$.
%It defaults to allow as many keys as the specified cache size will allow.

%It is expected that increasing the cache will increase performance up to a point where it is putting an undue burden on memory.

%This experiment will not vary the row cache size.
%Based on the description \cite{ConfiguringCaches}, the row cache option entertains a very specific type of repeated query where different columns of the same row would be of interest.
%It is the author's prediction that this will not add significant value compared to varying the key cache.

%It should be noted that these are the global parameters.
%Individual table parameters can be configured \cite{TableProperties} to either disable or enable the cache.
%However, the 

%\subsection{Leveled Compaction}

%This is another form of caching.
%In a sense, "Leveled Compaction" is assisted caching.
%Cassandra makes use of a the data structure \gls{sstable} \cite{LeveledDataStax} which is a critical part of its compaction strategy.
%Compaction is meant to increase read speed.
%This experiment will change among compaction strategies.
%This requires multiple nodes to be executing read and write instructions.

%\subsection{Bloom Filter}

%This is a value that tunes the cache, and it can range from 0.0 to 1.0, recommended setting being 0.1 \cite{TableProperties}.

%This experiment will vary the value of the bloom filter across its range.
%It is expected that for a given stress test, performance will peak at or around 0.1 and diminish as the value approaches the extremes.

%\subsection{Compression}

%Cassandra's compression  allows a user to explore a trade-off between decompression speed and compression effectiveness.

%\begin{table}
%\begin{tabular}{ | l | c | r | }
%  \hline
%  LZ4Compressor & decompresses fastest & least effective \\ \hline
%  SnappyCompressor (default) & - & - \\ \hline
%  DeflateCompressor & decompresses slowest & most effective \\ \hline
%  " " & no compression & no compression \\ \hline
%\end{tabular}
%\caption{Compression Algorithms in Cassandra \cite{CompressionSubproperties}}
%\end{table}
%This experiment will vary the compression algorithms to gauge a set.
%It is expected that the Raspberry Pi will not deviate from the expected design behavior.
%This experiment will be a characterization experiment.

%\section{Setting Up the Experiment/Topology}


%\subsection{Wireless Data}


%\subsection{Topologies of Interest}







This paper explores distributed NoSQL database Cassandra’s performance limitations in an \gls{iot} using hardware with limited storage space and price, namely the Raspberry Pi 2.
Our aim is to use Cassandra’s reliable and efficient data distribution to enable distributed exploits on real-time streaming data.
At the time of this writing, the proposition of Cassandra on \gls{iot}-like hardware carries some development risk for a would-be application developer.
This work not only demonstrates that actual operation of Cassandra is possible on Raspberry Pi, but also varies the conditions of operation to serve the expectation management of the variations inherent in new, creative, and cutting-edge applications. 
To represent \gls{iot}, we vary the memory on a virtual machine among 1GB, 2GB, and 4GB as well as implement Cassandra on the Raspberry Pi platform.
This work demonstrates the feasibility and generates floor function from which one can gauge expected performance and potential sampling rates when porting a distributed database like Cassandra from powerful, stationary nodes to less powerful, but more flexible nodes.
This work uses the \gls{ycsb} not only for its popularity but to catalyze infusion of this research with existing future research that will fully characterize \gls{iot} in the realm of controlled, flexible and resilient data storage.


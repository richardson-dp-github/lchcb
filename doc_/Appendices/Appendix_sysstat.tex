% Appendix Template

\chapter{Excerpts from the SYSSTAT man page} % Main appendix title

\label{SYSSTAT} 

\setstretch{1.0}


\begin{minted}[breaklines]{text}

NAME

sar - Collect, report, or save system activity information.
SYNOPSIS

sar [ -A ] [ -B ] [ -b ] [ -C ] [ -D ] [ -d ] [ -F [ MOUNT ] ] [ -H ] [ -h ] [ -p ] [ -q ] [ -R ] [ -r [ ALL ] ] [ -S ] [ -t ] [ -u [ ALL ] ] [ -V ] [ -v ] [ -W ] [ -w ] [ -y ] [ --sadc ] [ -I { int [,...] | SUM | ALL | XALL } ] [ -P { cpu [,...] | ALL } ] [ -m { keyword [,...] | ALL } ] [ -n { keyword [,...] | ALL } ] [ -j { ID | LABEL | PATH | UUID | ... } ] [ -f [ filename ] | -o [ filename ] | -[0-9]+ ] [ -i interval ] [ -s [ hh:mm[:ss] ] ] [ -e [ hh:mm[:ss] ] ] [ interval [ count ] ]
DESCRIPTION

The sar command writes to standard output the contents of selected cumulative activity counters in the operating system. The accounting system, based on the values in the count and interval parameters, writes information the specified number of times spaced at the specified intervals in seconds. If the interval parameter is set to zero, the sar command displays the average statistics for the time since the system was started. If the interval parameter is specified without the count parameter, then reports are generated continuously. The collected data can also be saved in the file specified by the -o filename flag, in addition to being displayed onto the screen. If filename is omitted, sar uses the standard system activity daily data file (see below). By default all the data available from the kernel are saved in the data file.

The sar command extracts and writes to standard output records previously saved in a file. This file can be either the one specified by the -f flag or, by default, the standard system activity daily data file. It is also possible to enter -1, -2 etc. as an argument to sar to display data of that days ago. For example, -1 will point at the standard system activity file of yesterday.

Standard system activity daily data files are named saDD or saYYYYMMDD, where YYYY stands for the current year, MM for the current month and DD for the current day. They are the default files used by sar only when no filename has been explicitly specified. When used to write data to files (with its option -o), sar will use saYYYYMMDD if option -D has also been specified, else it will use saDD. When used to display the records previously saved in a file, sar will look for the most  recent of saDD and saYYYYMMDD, and use it.

Standard system activity daily data files are located in the /var/log/sa directory by default. Yet it is possible to specify an alternate location for them: If a directory (instead of a plain file) is used with options -f or -o then it will be considered as the directory containing the data files.

Without the -P flag, the sar command reports system-wide (global among all processors) statistics, which are calculated as averages for values expressed as percentages, and as sums otherwise. If the -P flag is given, the sar command reports activity which relates to the specified processor or processors. If -P ALL is given, the sar command reports statistics for each individual processor and global statistics among all processors.

You can select information about specific system activities using flags. Not specifying any flags selects only CPU activity. Specifying the -A flag selects all possible activities.

The default version of the sar command (CPU utilization report) might be one of the first facilities the user runs to begin system activity investigation, because it monitors major system resources. If CPU utilization is near 100 percent (user + nice + system), the workload sampled is CPU-bound.  

If multiple samples and multiple reports are desired, it is convenient to specify an output file for the sar command. Run the sar command as a background process. The syntax for this is:  
sar -o datafile interval count >/dev/null 2>&1 &

All data are captured in binary form and saved to a file (datafile). The data can then be selectively displayed with the sar command using the -f option. Set the interval and count parameters to select count records at interval second intervals. If the count parameter is not set, all the records saved in the file will be selected. Collection of data in this manner is useful to characterize system usage over a period of time and determine peak usage hours.
Note:   The sar command only reports on local activities.

OPTIONS

.....


-r [ ALL ]

Report memory utilization statistics. The ALL keyword indicates that all the memory fields should be displayed. The following values may be displayed:
kbmemfree 

Amount of free memory available in kilobytes.
kbmemused 

Amount of used memory in kilobytes. This does not take into account memory used by the kernel itself.
%memused 

Percentage of used memory.
kbbuffers 

Amount of memory used as buffers by the kernel in kilobytes.
kbcached 

Amount of memory used to cache data by the kernel in kilobytes.
kbcommit 

Amount of memory in kilobytes needed for current workload. This is an estimate of how much RAM/swap is needed to guarantee that there never is out of memory.
%commit 

Percentage of memory needed for current workload in relation to the total amount of memory (RAM+swap). This number may be greater than 100% because the kernel usually overcommits memory.
kbactive


Amount of active memory in kilobytes (memory that has been used more recently and usually not reclaimed unless absolutely necessary).
kbinact


Amount of inactive memory in kilobytes (memory which has been less recently used. It is more eligible to be reclaimed for other  purposes).
kbdirty

Amount of memory in kilobytes waiting to get written back to the disk.
kbanonpg

Amount of non-file backed pages in kilobytes mapped into userspace page tables.
kbslab

Amount of memory in kilobytes used by the kernel to cache data structures for its own use.
kbkstack

Amount of memory in kilobytes used for kernel stack space.
kbpgtbl

Amount of memory in kilobytes dedicated to the lowest level of page tables.
kbvmused

Amount of memory in kilobytes of used virtual address space.




\end{minted}